\documentclass{amsart}
%%%%%%%%%%%%%%%%%%%%%%%%%%%%%%%%%%%%%%%%%
%version 3 of paper

\usepackage{amscd}
\usepackage{amsmath}
\usepackage{amssymb}
\usepackage{amsthm}
\usepackage{subfig}
\usepackage{graphicx}
%\usepackage{geometry}
%\usepackage[all,cmtip]{xy}
\usepackage{tikz}
\usetikzlibrary{shapes.geometric, arrows}

%\setlength{\textheight}{9.8in} \setlength{\topmargin}{0.0in}
%\setlength{\headheight}{0.0in} \setlength{\headsep}{0.0in}
%\setlength{\leftmargin}{0.5in}\setlength{\oddsidemargin}{0.0in}
%\setlength{\parindent}{0pc}\linespread{1.6}
%\setlength{\textwidth}{6in}


%%%%%%%%%%%%%%%%%%%%%%%%%%%%%%%%%%%%%%%%%%%%%
\theoremstyle{plain}
\newtheorem{thm}{Theorem}[section]
\newtheorem{lem}[thm]{Lemma}
\newtheorem{prop}[thm]{Proposition}
\newtheorem{cor}[thm]{Corollary}
\newtheorem{rem}[thm]{Remark}
\newtheorem{alem}[thm]{Almost A Lemma}
\newtheorem{lts}[thm]{Left to Show}
\newtheorem*{dfn}{Definition}
\newtheorem*{conj}{Conjecture}



\newenvironment{poc}[1][Proof of Claim]{\begin{trivlist}
\item[\hskip \labelsep {\bfseries #1}]}{\end{trivlist}}



%%%%%%%%%%%%%%%%%%%%%%%%%%%%%%%%%%%%%%%%%%%%%


\newcommand{\GL}{\Gamma_L}
\newcommand{\GLone}{\Gamma_{L'}}
\newcommand{\Gnot}{\Gamma_0}
\newcommand{\GK}{\Gamma_K}
%\newcommand{\GDS}{\Gamma_{D_s}}
%\newcommand{\GDF}{\Gamma_{D_f}}
\newcommand{\GDS}{\Gamma_{s}}
\newcommand{\GDF}{\Gamma_{f}}


\newcommand{\NGL}{N(\Gamma_L)}
\newcommand{\SNL}{\Sth - L}
\newcommand{\CGL}{Comm^+(\Gamma_L)}
\newcommand{\Hth}{\mathbb{H}^3}
\newcommand{\PSLTC}{PSL(2, \mathbb{C})}
\newcommand{\PSLTF}{PSL(2, \mathbb{F})}
\newcommand{\PSLTZI}{PSL(2, \mathbb{Z}[i])}
\newcommand{\SLTC}{SL(2, \mathbb{C})}
\newcommand{\TT}{\mathcal{T}}
\newcommand{\Z}{\mathbb{Z}}
\newcommand{\ZnZ}{\mathbb{Z}/n\mathbb{Z}}
\newcommand{\ZthZ}{\mathbb{Z}/2\mathbb{Z}}
\newcommand{\ZFZ}{\mathbb{Z}/4\mathbb{Z}}
\newcommand{\Q}{\mathbb{Q}}
\newcommand{\Qy}{\mathbb{Q}(y)}


\newcommand{\TtwoxI}{\mathbb{T}^2 \times I}
\newcommand{\PxS}{\mathbb{P} \times \mathbb{S}^1}
\newcommand{\RPth}{RP^3}
\newcommand{\Sth}{S^3}
\newcommand{\SxD}{\mathbb{S}^1 \times D}
\newcommand{\SxStwo}{\mathbb{S}^1 \times S^2}


\begin{document}
\title{Distance from the center of an ideal tetrahedron to a side}
\author{G\"{o}rner, Haraway, Hoffman, Trnkova}
\date{\today}
\maketitle

Let $\Delta$ be an ideal hyperbolic tetrahedron.
$\Delta$'s group of symmetries has a
unique fixed point, the \emph{center} $p_\Delta$. Generically
this group is a Klein four-group consisting of the identity
and half-revolutions about the perpendiculars between
opposite pairs of edges. The cones off the faces of $\Delta$ to $p_\Delta$ are
isometric to one another (via the Klein four-group).
Let $f$ be a face of $\Delta$. The cone $C_f$ off $f$ to $p_\Delta$
consists of the points in $\Delta$ closest to $f$. Therefore,
to find the point in $\Delta$ of maximum injectivity radius,
it will suffice to find the point $p_f$ in $C_f$ farthest from $f$,
and to calculate the distance from $p_f$ to $f$. In fact we suspect
of course that $p_f$ is $p_\Delta$. So instead of calculating the
maximal injectivity radius, we calculate the distance from $p_\Delta$
to $f$, and hope later to show that $p_\Delta$ realizes the maximal
injectivity radius.

Because we intend to work in SnapPy, we work with Thurston's
parametrization of the isometry classes of ideal hyperbolic
tetrahedra. For convenience we assume our tetrahedron comes
with a choice of three of its vertices in order. Identifying
the boundary of $H^3$ with $\mathbf{CP}^1$, when we take
these three vertices respectively to $0,1$ and $\infty$ via
an isometry, the remaining vertex is taken to some element
$z \in \mathbf{C}\setminus\{0,1\}$. In fact, to reduce
the number of cases, and intending to work only with
geometric solutions to the gluing equations, we will assume
that this element $z$ is in $\mathcal{H}$, the upper-half plane.

With these assumptions---i.e. three vertices at 0, 1, and $\infty$,
and the other $z \in \mathcal{H}$---we may begin calculation.
We will calculate $p_\Delta$ as a point in upper-half space,
then calculate its distance to the plane through 0, 1, and $\infty$.

Now, $p_\Delta$ is the fixed point of three involutions.
It will suffice to find the intersection of the axes of
two of these involutions. The simplest of these are
the two involutions effecting the two permutations 
$(\infty\, 1)(0\, z)$ and $(\infty\, 0)(1\, z)$.
These involutions, you may verify, are given
by $M_0.w = (w-z)/(w-1)$ and $M_1.w = z/w$.

The axes of these involutions are their fixed points.
We can determine these axes by their (fixed) endpoints,
and we can solve for these endpoints. You can verify
that $M_0.w = w$ when $w = 1 \pm \sqrt{1-z}$, and
$M_1.w = w$ when $w = \pm \sqrt{z}$.

We can represent upper-half-space as $\mathbf{C} \times \mathbf{R_+}$.
The projection of the axis of $M_0$ to $\mathbf{C}$ is
the line segment between $\lambda_+$ and $\lambda_1$, its
endpoints, and likewise for $M_1$. 

The $\mathbf{C}$-coordinate $W$
of $p_\Delta$ is the intersection 
of these line segments. $W$ is that 
complex number such that there exist
$\lambda, \mu \in [-1,1]$ such that 
both $W = \lambda \cdot \sqrt{z}$
and $W = 1 + \mu \sqrt{1-z}$. We take
$\mathbf{H}$ as the codomain of the square-root.

The calculation proper now begins.
Choosing $i^2 = -1$ so that $H^3$
gets the usual right-hand orientation,
let $z = a+b\cdot i$ with $a, b \in \mathbf{R}$.
Then 
\[
\sqrt{z} = \sqrt{\frac{\|z\| - a}{2}} + i \cdot \sqrt{\frac{\|z\| + a}{2}}, \mbox{ and}
\]
\[
1+\sqrt{1-z} = 1 - \sqrt{\frac{\|1-z\| + 1 - a}{2}} + i \cdot \sqrt{\frac{\|1 - z\| + a - 1}{2}}.
\]
We have $\lambda \cdot \sqrt{z} = 1 + \mu \sqrt{1-z}$.
Splitting this into real and imaginary parts yields
\begin{align*}
\lambda \cdot \sqrt{\frac{\|z\| + a}{2}} &= 1 - \mu \cdot \sqrt{\frac{\|1-z\| + 1 - a}{2}},\\
\lambda \cdot \sqrt{\frac{\|z\| - a}{2}} &= \mu \cdot \sqrt{\frac{\|1-z\| + a - 1}{2}}.
\end{align*}
We can solve this to get
\begin{align*}
\lambda &= \frac{\sqrt{2} \cdot \sqrt{\|1-z\| + a-1}}
            {(\sqrt{\|1-z\|+1-a}\cdot \sqrt{\|z\|-a})+(\sqrt{\|1-z\|+a-1}\cdot\sqrt{\|z\|+a})},\\
\mu &= \frac{\sqrt{2} \cdot \sqrt{\|z\| - a}}
            {(\sqrt{\|1-z\|+1-a}\cdot \sqrt{\|z\|-a})+(\sqrt{\|1-z\|+a-1}\cdot\sqrt{\|z\|+a})},
\end{align*}
so that
\begin{equation}
W = \frac{\sqrt{\|1-z\| + a-1}\cdot (\sqrt{\|z\|+a} + i \cdot \sqrt{\|z\|-a})}
            {(\sqrt{\|1-z\|+1-a}\cdot \sqrt{\|z\|-a})+(\sqrt{\|1-z\|+a-1}\cdot\sqrt{\|z\|+a})}
\end{equation}
We may need later to manipulate this 
equation into a numerically more 
stable form, but for now we leave it as is.

To calculate the height of $p_\Delta$ we
just consider that $p_\Delta$ lies on the
line between $\sqrt{z}$ and $-\sqrt{z}$.
In the upper-half-plane model this line is
the Euclidean half-circle
$\{(w,z) : t>0,\ t^2 + \|w\|^2 = \|z\|\}$.
Clearly then $p_\Delta = (W, \sqrt{\|z\| - \|W\|^2}) = (W, \sqrt{1-\lambda^2}\cdot\|z\|)$.
This too might need some manipulation
before implementing it in code. As a
first step towards such manipulation,
note that, letting 
$X = \sqrt{\|z\|-a}/\sqrt{\|1-z\| + a-1}$, we can rewrite
\begin{equation}\label{eqn:lmb}
\lambda = \frac{\sqrt{2}}
            {X\cdot \sqrt{\|1-z\|+1-a}
             +\sqrt{\|z\|+a}}.
\end{equation}
Using L'H\^{o}pital's rule shows that as the
imaginary part of $z$ goes to 0, $X^2$ goes
to $(1-a)/a$, and hence $X$ goes to $\sqrt{(1-a)/a}$,
which is well-defined and nonzero away from $a = 0, 1$.
If we implement the function $z \mapsto X$ so that
it is numerically stable and well-defined away
from $a = 0,1$, then we've defined $\lambda$ much
more nicely, and consequently $p_\Delta$ too.



Having calculated $p_\Delta$, we can
now conclude our discussion by calculating
its distance to the plane through $0,1,\infty$,
the plane $\mathbf{R} \times \mathbf{R}_+ \subset \mathbf{C} \times \mathbf{R}_+$.
The 
\end{document}
